\section{HSS with Certified Deletion}\label{sec:hss-cd}

Unless otherwise specified, we will consider the following kind of
HSS schemes by default:
\begin{itemize}
\item Are 2-out-of-2 secret-sharing schemes.
\item Allow evaluation for a single secret.
\end{itemize}

An HSS scheme with certified deletion must have the following syntax
and correctness requirements:

\subsection{HSS-CD Syntax}
A scheme satisfying the HSS-CD syntax for a PPT circuit family
$\calC$ is a tuple of 5 algorithms $\HSScd = \HSScd.(\Sh, \Ev,$
$\Del, \Vrfy, \Rec)$ with the following properties:

\begin{description}

\item [Syntax:] $ $
\item $\Sh(s) \rightarrow (\qsh_0^0, \vk_0), (\qsh_1^0,
\vk_1)$: The sharing algorithm outputs quantum
(possibly-entangled) secret-shares $\qsh_0^0, \qsh_1^0$ encoding an
input secret $s$. It also outputs the corresponding classical
verification keys $\vk_0, \vk_1$.

\item $\Ev(C_j, i, \qsh_i^{j-1}) \rightarrow \qsh_i^{j}$: The
evaluation algorithm takes the description of a PPT computable
circuit $C_j$, an index $i \in \bit$, and an input share
$\qsh_i^{j-1}$. It outputs a possibly-altered output share
$\qsh_i^{j}$.

\item $\Del(i, \qsh_i^j) \rightarrow \cert_i$: The deletion
algorithm takes an index $i \in \bit$, a corresponding share
$\qsh_i^j$, and produces a deletion certificate $\cert_i$.

\item $\Vrfy(i, \vk_i, \cert_i) \rightarrow \top/\bot$: The
verification algorithm takes an index $i \in \bit$, the
corresponding verification key $\vk_i$ and a certificate $\cert_i$.
It outputs $\top$ or $\bot$.

\item $\Rec(\qsh_0^q, \qsh_1^q) \rightarrow (d_1, \cdots, d_q)$: The
reconstruction algorithm takes two evaluated input shares $\qsh_0^q,
\qsh_1^q$ and outputs a $q$-tuple $(d_1, \cdots, d_q)$.

\item [Evaluation Correctness:] $\forall q = \poly(\secp)$
and $\forall$ $(C_1, \ldots, C_q) \in \mathcal{C}^{q}$, the
following condition holds:
\begin{align}
\Pr\left[
(d_1, \cdots, d_q) = (C_1(s), \cdots, C_q(s))
\ :
\begin{array}{ll}
(\qsh_0^0, \vk_0), (\qsh_1^0, \vk_1) \gets \Sh(s)\\
\forall i,j \in \bit \times [q]: \qsh_i^j \leftarrow \Ev(C_j, i,
\qsh_i^{j-1})\\
(d_1, \cdots, d_q) \leftarrow \Rec(\qsh_0^q, \qsh_1^q)
\end{array}
\right] \geq 1 - \negl(\secp)
\end{align}

\item [Deletion Correctness:] The following condition holds for
all $i \in \bit$, $q = \poly(\secp)$ and $(C_1, \ldots, C_q) \in
\mathcal{C}^q:$
\begin{align}
\Pr\left[
\Vrfy(i, \vk_i, \cert_i) \rightarrow \top
\ :
\begin{array}{ll}
(\qsh_0^0, \vk_0), (\qsh_1^0, \vk_1) \gets \Sh(s)\\
\forall j \in [q]: \qsh_i^j \leftarrow \Ev(C_j, i,
\qsh_i^{j-1})\\
\cert_i \leftarrow \Del(i, \qsh_i^q)
\end{array}
\right] \geq 1 - \negl(\secp)
\end{align}

\item [Compactness:] The following condition holds for all $i \in
\bit$, $q = \poly(\secp)$ and $(C_1, \ldots, C_q) \in
\mathcal{C}^q$, where $l_q$ denotes the output length of the circuit
$C_q$:

\begin{align}
\left[
|\qsh_i^q| - |\qsh_i^{q-1}| = \poly(1^\secp, l_q)
\ :
\begin{array}{ll}
(\qsh_0^0, \vk_0), (\qsh_1^0, \vk_1) \gets \Sh(s)\\
\forall j \in [q]: \qsh_i^j \leftarrow \Ev(C_j, i,
\qsh_i^{j-1})\\
\end{array}
\right]
\end{align}
\end{description}

\subsection{Additive HSS-CD Syntax}

A scheme satisfying the additive HSS-CD syntax for a PPT circuit
family $\calC$ is a tuple of 5
algorithms $\HSScd = \HSScd.(\Sh, \Ev, \Obs, \Del, \Vrfy)$ with the
following properties:

\begin{description}

\item [Syntax:] $ $
\item $\Sh(s) \rightarrow (\qsh_0^0, \vk_0), (\qsh_1^0,
\vk_1)$: The sharing algorithm outputs quantum
(possibly-entangled) secret-shares $\qsh_0^0, \qsh_1^0$ encoding an
input secret $s$. It also outputs the corresponding classical
verification keys $\vk_0, \vk_1$.

\item $\Ev(C_j, i, \qsh_i^{j-1}) \rightarrow \qsh_i^{j}$: The
evaluation algorithm takes the description of a PPT computable
circuit $C_j$, an index $i \in \bit$ and a share
$\qsh_i^{j-1}$. It outputs a quantum share $\qsh_i^j$.

\item $\Obs(i, \qsh_i^j) \rightarrow (y_0^1, \ldots, y_0^j):$ The
observation algorithm takes an index $i \in \bit$ and a quantum
state $\qsh_i^j$ and produces a $j$-tuple of classical shares.

\item $\Del(i, \qsh_i^j) \rightarrow \cert_i$: The deletion
algorithm takes an index $i \in \bit$, a corresponding quantum share
$\qsh_i^j$, and produces a deletion certificate $\cert_i$.

\item $\Vrfy(i, \vk_i, \cert_i) \rightarrow \top/\bot$: The
verification algorithm takes an index $i \in \bit$, the
corresponding verification key $\vk_i$ and a certificate $\cert_i$.
It outputs $\top$ or $\bot$.

\item [Evaluation Correctness:] The following
condition holds for all $q = \poly(\secp)$ and $(C_1, \ldots, C_q)
\in \mathcal{C}^q$:
\begin{align}
\Pr\left[
(y_0^1 \xor y_1^1, \cdots, y_0^q \xor y_1^q) = (C_1(s), \cdots,
C_q(s))
\ :
\begin{array}{ll}
(\qsh_0^0, \vk_0), (\qsh_1^0, \vk_1) \gets \Sh(s)\\
\forall i,j \in \bit \times [q]: \qsh_i^j
\leftarrow \Ev(C_j, i,
\qsh_i^{j-1})\\
(y_i^1, \ldots, y_i^q) \leftarrow \Obs(i, \qsh_i^q)
\end{array}
\right] \geq 1 - \negl(\secp)
\end{align}

In the case of additive HSS-CD, we will consider the following
weaker deletion guarantee: 

\item [Deletion Correctness:] The following condition holds for
all $i \in \bit$, $q = \poly(\secp)$ and $(C_1, \ldots, C_q) \in
\mathcal{C}^q:$
\begin{align}
\Pr\left[
\Vrfy(i, \vk_i, \cert_i) \rightarrow \top
\ :
\begin{array}{ll}
(\qsh_0^0, \vk_0), (\qsh_1^0, \vk_1) \gets \Sh(s)\\
\forall j \in [q]: \qsh_i^j \leftarrow \Ev(C_j, i,
\qsh_i^{j-1})\\
\cert_i \leftarrow \Del(i, \qsh_i^q)
\end{array}
\right] \geq 1 - \negl(\secp)
\end{align}
\end{description}



\subsection{Security Definitions}
\paragraph{Statistical/Computational Deletion Security wrt Share $j$:}
The following security notion is defined wrt a non-local quantum
adversary $(\qA_0, \qA_1)$:

\begin{description}
\item [$\expa{\HSScd, (\qA_0, \qA_1)}{del}(1^\secp, j, b)$:] $ $
\begin{enumerate}
\item $\qA_0$ sends $(s_0, s_1) \in \bit^\secp$ to the challenger.
\item 
The challenger runs
$(\qsh_0^0, \vk_0), (\qsh_1^0, \vk_1) \gets \Sh(s_b)$ and sends each
$\qsh_i^0$ to party $P_i$.

\item $\qA_j$ sends $(\cert_j, R_j)$ and $\qA_{1-j}$ sends $R_{1-j}$
where $R_0, R_1$ are some registers.
\item If $\Vrfy(j, \vk_j, \cert_j) = \top$, then output $(R_1, R_2)$.
\end{enumerate}

Statistical Deletion Security wrt Share $j$ holds if the following
holds:

$$TD\Big(\expa{\HSScd, (\qA_0, \qA_1)}{del}(1^\secp, j,
0), \expa{\HSScd, (\qA_0, \qA_1)}{del}(1^\secp, j,
1)\Big) \le \negl(\secp)$$

Computational Deletion Security wrt Share $j$ holds if the following
holds for all QPT $\qA$:

$$\bigg\lvert \Pr\Big[\qA\Big(\expa{\HSScd, (\qA_0,
\qA_1)}{del}(1^\secp, j, 0)\Big)=1\Big] -
\Pr\Big[\qA\Big(\expa{\HSScd, (\qA_0,
\qA_1)}{del}(1^\secp, j, 1)\Big)=1\Big]
\bigg\rvert \le \negl(\secp)$$

\end{description}

\paragraph{Statistical/Computational Double-Deletion Security:}

The following security notion is defined wrt a non-local quantum
adversary $(\qA_0, \qA_1)$:

\begin{description}
\item [$\expb{\HSScd, (\qA_0, \qA_1)}{del}{2}(1^\secp, b)$:] $ $
\begin{enumerate}
\item 
\item $\qA_0$ sends $(s_0, s_1) \in \bit^\secp$ to the challenger.
The challenger runs
$(\qsh_0^0, \vk_0), (\qsh_1^0, \vk_1) \gets \Sh(s_b)$ and sends each
$\qsh_i^0$ to party $P_i$.

\item $\qA_0$ sends $(\cert_0, R_0)$ and $\qA_{1}$ sends
$(\cert_{1}, R_{1})$ where $R_0, R_1$ are some registers.
\item If $\Vrfy(0, \vk_0, \cert_0) = \Vrfy(1, \vk_1, \cert_1)=
\top$, then output $(R_0, R_1)$.
\end{enumerate}

Statistical Double-Deletion Security holds if the following holds:

$$TD\Big(\expb{\HSScd, (\qA_0, \qA_1)}{del}{2}(1^\secp, 
0), \expb{\HSScd, (\qA_0, \qA_1)}{del}{2}(1^\secp,
1)\Big) \le \negl(\secp)$$

Computational Double-Deletion Security holds if the following holds
for all QPT $\qA$:

$$\bigg\lvert \Pr\Big[\qA\Big(\expb{\HSScd, (\qA_0,
\qA_1)}{del}{2}(1^\secp, 0)\Big)=1\Big] -
\Pr\Big[\qA\Big(\expb{\HSScd, (\qA_0,
\qA_1)}{del}{2}(1^\secp, 1)\Big)=1\Big]
\bigg\rvert \le \negl(\secp)$$

\end{description}

Hereafter, we will use \emph{stat} to denote statistical
security and \emph{comp} to denote computational security.

\begin{definition}[(Additive) (X, Y)-HSS-CD scheme for $\calC$]
An (Additive) (X, Y)-HSS-CD scheme for $\calC$, where X, Y $\in
\{\textrm{stat},
\textrm{comp}\}^2$  is a scheme that satisfies the (Additive) HSS-CD
syntax for $\calC$, the X deletion security for share $0$, and the
Y deletion security for share $1$.
\end{definition}

\begin{definition}[(Additive) (X)-HSS-CD scheme] An (Additive)
(X)-HSS-CD scheme for $\calC$ where $X \in \{\textrm{stat},
\textrm{comp}\}$ is a scheme satisfying the (Additive) HSS-CD syntax
for $\calC$ and the X double-deletion security.
\end{definition}

\begin{remark}
Observe that a (stat, comp)-HSS-CD scheme for $\calC$ is also
a (stat)-HSS-CD scheme for $\calC$. Likewise, a (comp,
comp)-HSS-CD scheme for $\calC$ is also a (comp)-HSS-CD scheme for
$\calC$.
\end{remark}

\section{Impossibility Results}

\begin{lemma}
Any (stat, stat)-HSS-CD scheme for $\cal{C}$ is also an
information-theoretic HSS scheme for $\calC$.
\end{lemma}
\begin{proof}
Suppose there exists a (stat, stat)-HSS-CD scheme that does not
satisfy information-theoretic secrecy. Then, there exists an
unbounded adversary $\qD$ that receives some share $\qsh_i^0$ and
distinguishes between the secrets $s_0, s_1$. Then, there exists an
adversary $(\qA_0, \qA_1)$ in the statistical security wrt share
$1-i$ game that works as follows. $\qA_{1-i}$ honestly deletes its
share and outputs a dummy register while $\qA_i$ outputs a register
containing its share $\qsh_i^0$. In the second-stage, the
distinguisher $\qD$ is run on $\qsh_i^0$ to tell apart the secrets
$s_0, s_1$.
\end{proof}

The following theorem shows that the classical impossibility result
regarding information-theoretic HSS by \cite{ITCS:BGILT18} also
applies to the setting of quantum shares.

\begin{theorem}
TBD.
\end{theorem}

\begin{theorem}
There does not exist an Additive (comp)-HSS-CD scheme $\HSScd$ for
any PPT circuit class $\calC$, 
given that $\HSScd.\Sh(s)$ outputs shares $\qsh_0, \qsh_1$ that are
not entangled with each other.
\end{theorem}
\begin{proof}
In fact, we will prove that this holds even for a weaker notions of
evaluation and deletion correctness, where $\Ev$ and $\Del$ support
only a single evaluation. Specifically, for shares $(\qsh_0,
\qsh_1)$ output by $\Sh(s)$, $\Ev(i, C, \qsh_i)$ outputs a share
$\nqsh_i$, and $\Obs(i, \nqsh_i)$ outputs a value $y_i$ such that
$y_0 \xor y_1 = C(s)$. Moreover, $\Del(i, \nqsh_i)$ outputs
$\cert_i$ such that $\Vrfy(i, \vk_i, \cert_i) = \top$. The argument
proceeds as follows:\\

Let $(\ket{\psi_0}, \vk_0), (\ket{\psi_1}, \vk_1)$ be some pure
state output by $\Sh(s)$, where $\ket{\psi_0}, \ket{\psi_1}$ are not
entangled with each other. Let $\ket{\widetilde{\psi_i}}$ be the
state output by $\Eval(i, C, \ket{\psi_i})$.
Wlog, let $\{\Pi_0, \mathbb{I} - \Pi_0\}$
be the projective measurement equivalent of $\Obs(0,
\ket{\widetilde{\psi_0}})$, i.e., $\Pi_0$ corresponds to $y_0 = 0$
and $\mathbb{I} - \Pi_1$ corresponds to $y_0 = 1$. Let $Y_0$ denote
the random variable of the output $y_0$. Likewise, consider the
projective measurement $\{\Pi_1, \mathbb{I} - \Pi_1\}$ equivalent of
$\Obs(1, \ket{\widetilde{\psi_1}})$ and let $Y_1$ be the
corresponding output
random variable. Notice that for every outcome $y_0$ of $Y_0$, there
is a single outcome $y_1$ of $Y_1$ that satisfies $y_1 = C(s) \xor
y_0$. Let $\widetilde{Y}_0$ be the random variable for
$\widetilde{y_0}$ sampled as $\widetilde{y_0} = C(s) \xor y_1 : y_1
\leftarrow Y_1$. By the evaluation correctness requirement, we
require that $\Pr[Y_0 = \widetilde{Y_0}] \ge 1 - \negl(\secp)$.
Since $Y_0$ and $\widetilde{Y_0}$ are independent random variables,
this is only possible if there exists $y_0^\star$ such that $\Pr[Y_0
= y_0^\star] \geq 1 - \negl(\secp)$ and $\Pr[\widetilde{Y_0} =
y_0^\star] \geq 1 - \negl(\secp)$. In other words, the measurement
$\{\Pi_0, \mathbb{I} - \Pi_0\}$ either accepts the state
$\ket{\psi_0}$ with probability $1 - \negl(\secp)$ or rejects it
with probability $1 - \negl(\secp)$. Consequently, by the gentle
measurement lemma, the leftover state is close in trace distance to
the state $\ket{\psi_0}$. As a result, it can be certifiably deleted
after obtaining $y_0$. By a similar argument, $y_1$ can be obtained
in the same way. Since this holds for every possible pure state
output by $\Sh(s)$, it also holds for arbitrary mixed states. As a
result, the adversary can efficiently compute $y_0 \xor y_1 = C(s)$
in the second-stage, breaking the computational double-deletion
security. Since this security notion is the weakest one, this also
rules out the other notions.  \end{proof}

\section{Feasibility Results}

\subsection{FHE-CD based Construction}

We construct a (stat, comp)-HSS-CD scheme
$\HSScd = \HSScd.(\Sh, \Ev, \Del, \Vrfy, \Rec)$ using the following
building blocks.

\begin{itemize}
\item Fully Homomorphic Encryption with Certified Deletion (FHE-CD)
scheme $\FHEcd = \FHEcd.(\Setup, \qEnc, \qDec, \\
\Ev, \Del, \Vrfy)$.

\item Secret Sharing with Certified Deletion (SS-CD) scheme $\SScd =
\SScd.(\Sh, \Rec, \Del, \Vrfy)$.
\end{itemize}

The construction is as follows.

\begin{description}

\item[$\HSScd.\Sh(s)$:] $ $
\begin{enumerate}
    \item Generate $(\pk, \sk) \leftarrow \FHEcd.\Setup(1^\secp)$.
    \item Compute $(\fhecd.\qct^0, \fhecd.\vk) \leftarrow
        \FHEcd.\qEnc(s)$.
    \item Compute $(\sscd.\qsh, \sscd.\csh), \sscd.\vk
    \leftarrow \SScd.\Sh(\sk)$.
\item Set $\qsh_0^0 \seteq (\fhecd.\pk, \fhecd.\qct^0, \sscd.\csh)$
and $\vk_0 \seteq \fhecd.\vk$.
\item Set $\qsh_1^0 \seteq \sscd.\qsh$ and $\vk_1 \seteq \sscd.\vk$.
\item Output $(\qsh_0^0, \vk_0), (\qsh_1^0, \vk_1)$.
\end{enumerate}

\item[$\HSScd.\Ev(C_j, i, \qsh_i^{j-1})$:] If $i = 1$, set
$\qsh_1^{j} \seteq \qsh_1^{j-1}$. Else, execute the following:
\begin{enumerate}
\item Parse $\qsh_{0}^{j-1}$ as $(\fhecd.\pk, \fhecd.\qct^{j-1},
\sscd.\csh)$.
\item Compute $\fhecd.\qct^j \leftarrow \FHEcd.\Ev(\fhecd.\pk, C_j,
\fhecd.\qct^{j-1})$
\item Set $\qsh_0^{j} \seteq (\fhecd.\pk, \fhecd.\qct^{j-1}
\sscd.\csh)$.
\item Output $\qsh_i^j$.
\end{enumerate}

\item[$\HSScd.\Del(i, \qsh_i^j)$:] $ $
\begin{enumerate}
\item If $i = 0$, execute the following:
\begin{enumerate}[(i)]
\item Parse $\qsh_0^j$ as $(\fhecd.\pk, \fhecd.\qct^j, \sscd.\csh)$.
\item Compute and output $\cert_0 \leftarrow
\FHEcd.\Del(\fhecd.\qct^j)$.
\end{enumerate}
\item If $i = 1$, execute the following:
\begin{enumerate}[(i)]
\item Parse $\qsh_1^j$ as $\sscd.\qsh$.
\item Compute and output $\cert_1 \leftarrow
\SScd.\Del(\sscd.\qsh)$.
\end{enumerate}
\end{enumerate}

\item[$\HSScd.\Vrfy(i, \vk_i, \cert_i)$:] $ $

\begin{enumerate}
\item If $i = 0$, output $\ans_0 \leftarrow \FHEcd.\Vrfy(\vk_0,
\cert_0)$.
\item If $i = 1$, output $\ans_0 \leftarrow \SScd.\Vrfy(\vk_1,
\cert_1)$.
\end{enumerate}

\item[$\HSScd.\Rec(\qsh_0^q, \qsh_1^q)$:] $ $

\begin{enumerate}
\item Parse $\qsh_0^q$ as $(\fhecd.\pk, \fhecd.\qct^q, \sscd.\csh)$.
\item Parse $\qsh_1^q$ as $\sscd.\qsh$.
\item Compute $\sk \leftarrow \SScd.\qDec(\sscd.\qsh, \sscd.\csh)$.
\item Compute and output $(d_1, \ldots, d_q) \leftarrow
\FHEcd.\qDec(\sk, \fhecd.\qct^q)$.

\end{enumerate}


\end{description}

\begin{theorem}
There exists a (stat, comp)-HSS-CD scheme assuming the existence of
a fully homomorphic encryption scheme with certified deletion
(FHE-CD), and a secret-sharing scheme with certified deletion
(SS-CD).
\end{theorem}

\begin{proof}
We will prove that the construction $\HSScd$ is a (stat,
comp)-HSS-CD scheme. First, we will assume that $(\qA_0, \qA_1)$ is
a non-local adversary that breaks the statistical deletion
security of share $0$. We will use this adversary to break the
certified deletion security of the FHE-CD scheme $\FHEcd$. Consider
a QPT reduction $\qR$ that runs as follows in the $\FHEcd$ game:

\begin{description}
\item Execution of $\qR^{(\qA_0, \qA_1)}$ in
$\expb{\FHEcd,\qR}{fhe}{cd}(1^\secp,b)$:
\begin{enumerate}
\item $\qA_0$ sends $(s_0, s_1) \in \bit^\secp$ to $\qR$, which
$\qR$ forwards to the challenger.
\item The challenger samples $(\pk, \sk) \leftarrow
\Setup(1^\secp)$ and sends $\pk$ to $\qR$.
\item The challenger encrypts $s_b$ as $\qct \leftarrow \qEnc(\pk,
s_b)$ and sends $\qct$ to $\qR$.
\item $\qR$ computes $\qsh_0^0 \seteq (\pk, \qct, \sscd.\csh)$,
where $\sscd.\csh \leftarrow \SScd.\Sim(1^\secp)$.
\item $\qR$ runs $\qA_0$ on input $\qsh_0^0$. If $\qA_0$ outputs
($\cert_0$, $R_0$), $\qR$ sends $\cert_0$ to the challenger.
\item The challenger computes $\ans \leftarrow \Vrfy(\vk, \cert_0)$.
If $\ans = \top$, it sends $\sk$ to $\qR$. Else, it outputs $\bot$.
\item $\qR$ computes $\sscd.\qsh$ conditioned on $(\sscd.\qsh,
\sscd.\sh)$ encoding $\sk$.
\item $\qR$ sends $\qsh_1^0 \seteq \sscd.\qsh$ to $\qA_1$. If
$\qA_1$ outputs $R_1$, send $(R_0, R_1)$ to the challenger.
\end{enumerate}
\end{description}

We will now argue that if $(\qA_0, \qA_1)$ break statistical
security wrt share 0, then $\qR$ breaks the certified-deletion
security of $\FHEcd$. Observe that the view of $\qA_0$ in the
reduction is identically distributed to its view in $\expa{\HSScd,
(\qA_0, \qA_1)}{del}(1^\secp, 0, b)$. Now, notice that if
$\HSScd.\Vrfy(0, \vk_0, \cert_0)$ passes, then $\FHEcd.\Vrfy(\vk,
\cert_0)$ also passes.  Consequently, $\qR$ receives the secret key
$\sk$. By the information-theoretic secrecy of the scheme $\SScd$,
the view of $\qA_1$ is identically distributed to that in the
original experiment. As a result, $(R_0, R_1)$ are identically
distributed to that of the $\HSScd$ game. By assumption, there
exists an unbounded algorithm that can use $(R_0, R_1)$ to guess $b$
with non-negligible probability. This breaks the certified-deletion
security of $\FHEcd$.

Next, we will assume that $(\qA_0, \qA_1)$ is a non-local adversary
that breaks the computational deletion security of share $1$. We
will use this adversary to break the certified deletion security of
the SS-CD scheme $\SScd$. Consider a non-local reduction $(\qR_0,
\qR_1)$ that runs as follows:

\begin{description}
\item Execution of $(\qR_0^{\qA_0}, \qR_1^{\qA_1})$ in
$\expb{\SScd,(\qR_0, \qR_1)}{ss}{cd}(1^\secp,b)$:
\begin{enumerate}
\item $\qR_0$ samples $(\pk, \sk) \leftarrow
\FHEcd.\Setup(1^\secp)$. It sets $s_0 \seteq 0^\secp$ and $s_1
\seteq \sk$ and sends $(s_0, s_1)$ to the challenger.
\item The challenger computes $(\qsh, \csh, \vk) \leftarrow
\Sh(s_b)$. It sends $\csh$ to $\qR_0$ and $\qsh$ to $\qR_1$.
\item $\qR_0$ runs $\qA_0$. $\qA_0$ sends $(s'_0, s'_1)$ to $\qR_0$.
\item $\qR_0$ sends $(\pk, \qct, \csh)$ to $\qA_0$, where
$\qct \leftarrow \FHEcd.\qEnc(\pk, s'_c)$ and $c \leftarrow \bit$.
\item $\qA_0$ sends $R_0'$ to $\qR_0$. $\qR_0$ sets $R_0 \seteq
(R_0', c)$ and sends it to the challenger.
\item $\qR_1$ runs $\qA_1$ on input $\qsh$. If $\qA_1$ outputs
($\cert_1$, $R_1'$), then $\qR_1$ sets $R_1 \seteq R_1'$ and sends
it to the challenger.
\item The challenger computes $\ans = \Vrfy(\vk, \cert_1)$. If $\ans
= \top$, it outputs $(R_0, R_1)$.
\end{enumerate}
\end{description}

Consider now the experiment $\expb{\SScd,(\qR_0,
\qR_1)}{ss}{cd}(1^\secp,0)$. Notice that if there exists a
QPT algorithm $\qA$ that obtains the registers $(R_0', R_1')$ and
outputs $c' = c$ with probability $\frac12 + \nonnegl(\secp)$, then
the security of $\FHEcd$ is broken. This is because a reduction can
obtain an $\FHEcd$ ciphertext and simulate the view os $\qA_0,
\qA_1$ as needed, because knowledge of $\sk$ is not required.

By assumption, there exists a QPT algorithm $\qA$ that obtains
$(R_0', R_1')$ and outputs $c'=c$ with probability $\frac12 +
\nonnegl(\secp)$ in the experiment $\expb{\SScd,(\qR_0,
\qR_1)}{ss}{cd}(1^\secp,1)$.

Now, consider an algorithm $\qR$ that obtains $(R_0 = (c, R_0'),
R_1 = R_1')$. It runs $\qA$ on $(R_0', R_1')$ and checks if the
value $c'$ equals $c$ or not. If it is, then $\qR$ outputs $b'=1$,
otherwise it outputs $b'=0$. Consequently, $\qR$ outputs $b'=b$ with
probability $\frac12 + \nonnegl(\secp)$, breaking the security of
the scheme $\SScd$. This gives us a contradiction.
\end{proof}

\subsection{Spooky-Encryption based Construction with Entangled
Shares}

\section{HSS and FSS with Weak Certified-Deletion}

