\section{Introduction}\label{sec:motive}

In Secure Software Leasing (SSL) with finite-term security, a
function can be leased to a server such that the server is able to
evaluate the function an unbounded polynomially many times.
Additionally, the server can later be asked to revoke its lease in a
way that passes certain verification. It is guaranteed that upon
successful verification, the server no longer has the ability to
evaluate the function.  Clearly, if a function is learnable (it can
be replicated given only black-box access), then it cannot be
securely leased.  Consequently, one might hope to target the class
of unlearnable functions.  However, it was shown in the work of
Ananth and LaPlaca \cite{EC:AnaLaP21} that there exist unlearnable
functions for which this notion is impossible to achieve. In this
work, we explore the setting where a client can lease out a function
to two servers $S_0$ and $S_1$ in a distributed sense, such that
they can compute additive shares of evaluations of the function. We
will consider the following security notion:

\begin{description}
\item \textbf{Finite-Term Security:} This
security notion guarantees that the following events cannot both
happen:

\begin{enumerate}
\item Either server provides a certificate of revocation that
passes verification.
\item $S_0, S_1$ produce quantum states $\rho_0, \rho_1$ which are
given to a QPT adversary $\qA$. Given a random input $x^*$, $\qA$
produces the correct output of the software on $x^*$.
\end{enumerate}
\end{description}

Other than finite-term security, the stronger notions of
infinite-term security and copy-protection have also been studied in
the literature. These notions do not involve deletion certificates
and only require the adversary to produce two working copies of the
software. While infinite-term security requires the copies to work
with the honest evaluation algorithm, copy protection is even
stronger and permits arbitrary evaluations on the copies.  Later on,
we will also explore extensions of these notions to the distributed
setting.

\subsection{Motivation}

The motivation for the different aspects of the notion are as
follows:

\begin{itemize} \item \textbf{Additive Reconstruction:} The 
requirement of additive secret-sharing is highly desirable as
opposed to arbitrary (potentially even quantum) reconstruction.
Recall that in the case of SSL, a single party obtains evaluations
in the clear.  In applications where a server obtains different
software from several clients and needs to compute on the individual
outputs, it can do so easily. This is non-trivial in the distributed
setting. Additive secret-shares offer a middle ground by allowing
linear computations on the outputs for free. Moreover, they are in
the correct format for MPC protocols which can allow for more
general computations.  Furthermore, additive reconstruction enables
efficient reconstruction, both in terms of computation and
communication.

\item \textbf{Finite-Term Security:} It can be
reasonable to assume that the servers do not collude in the short
term, but over time they may collude or an adversary may
get access to their data. This security notion guarantees that such
a data breach from both servers does not grant the ability to
evaluate the software.

\item \textbf{Secrecy:} Apart from finite term security, we also
require that if the servers do not collude, they do not learn
anything about the leased program. This requirement is similar to
that of Function Secret Sharing (FSS). This provides a two-tier
security guarantee. For instance, the servers can be used to
delegate a decryption functionality without them being able to
decrypt themselves in real time. Although they may decrypt past
ciphertexts after collusion, that data may not be very relevant
later on.  \end{itemize}

Consider now the following example use case for this notion:

\paragraph{Use Case.} We will consider a delegation scenario where a
client needs to delegate a function $f$. Of course, it can use an
FHE ciphertext and send it to a single server but this complicates
the decryption procedure. Also, if the client's secret-key is leaked
at any point in the future, the function can be completely learnt.
More importantly though, multiple FHE ciphertexts cannot be
evaluated on and compressed easily, which can be especially
important when multiple clients are involved. One could also
consider Multi-Key FHE which would not only be computationally
expensive, but would need distributed decryption. In such a
scenario, Function Secret Sharing (FSS) can prove to be a  good
alternative due to the additive reconstruction property, albeit at
the cost of utilizing two servers.  However, there is now the
possibility that in the long term, the servers may be breached,
providing the adversary with the ability to evaluate the software.
Hence, DSL can also be viewed as an extension of FSS to the quantum
setting, where it is upgraded with an SSL like guarantee. 

