\section{FSS with Certified Deletion}\label{sec:fss-cd}

\subsection{Strong FSS}
A scheme satisfying the HSS-CD syntax for a PPT circuit family
$\calC$ is a tuple of 5 algorithms $\HSScd = \HSScd.(\Sh, \Ev,$
$\Del, \Vrfy, \Rec)$ with the following properties:

\begin{description}

\item [Syntax:] $ $
\item $\Sh(s) \rightarrow (\qsh_0^0, \vk_0), (\qsh_1^0,
\vk_1)$: The sharing algorithm outputs quantum
(possibly-entangled) secret-shares $\qsh_0^0, \qsh_1^0$ encoding an
input secret $s$. It also outputs the corresponding classical
verification keys $\vk_0, \vk_1$.

\item $\Ev(C_j, i, \qsh_i^{j-1}) \rightarrow \qsh_i^{j}$: The
evaluation algorithm takes the description of a PPT computable
circuit $C_j$, an index $i \in \bit$, and an input share
$\qsh_i^{j-1}$. It outputs a possibly-altered output share
$\qsh_i^{j}$.

\item $\Del(i, \qsh_i^j) \rightarrow \cert_i$: The deletion
algorithm takes an index $i \in \bit$, a corresponding share
$\qsh_i^j$, and produces a deletion certificate $\cert_i$.

\item $\Vrfy(i, \vk_i, \cert_i) \rightarrow \top/\bot$: The
verification algorithm takes an index $i \in \bit$, the
corresponding verification key $\vk_i$ and a certificate $\cert_i$.
It outputs $\top$ or $\bot$.

\item $\Rec(\qsh_0^q, \qsh_1^q) \rightarrow (d_1, \cdots, d_q)$: The
reconstruction algorithm takes two evaluated input shares $\qsh_0^q,
\qsh_1^q$ and outputs a $q$-tuple $(d_1, \cdots, d_q)$.

\item [Evaluation Correctness:] $\forall q = \poly(\secp)$
and $\forall$ $(C_1, \ldots, C_q) \in \mathcal{C}^{q}$, the
following condition holds:
\begin{align}
\Pr\left[
(d_1, \cdots, d_q) = (C_1(s), \cdots, C_q(s))
\ :
\begin{array}{ll}
(\qsh_0^0, \vk_0), (\qsh_1^0, \vk_1) \gets \Sh(s)\\
\forall i,j \in \bit \times [q]: \qsh_i^j \leftarrow \Ev(C_j, i,
\qsh_i^{j-1})\\
(d_1, \cdots, d_q) \leftarrow \Rec(\qsh_0^q, \qsh_1^q)
\end{array}
\right] \geq 1 - \negl(\secp)
\end{align}

\item [Deletion Correctness:] The following condition holds for
all $i \in \bit$, $q = \poly(\secp)$ and $(C_1, \ldots, C_q) \in
\mathcal{C}^q:$
\begin{align}
\Pr\left[
\Vrfy(i, \vk_i, \cert_i) \rightarrow \top
\ :
\begin{array}{ll}
(\qsh_0^0, \vk_0), (\qsh_1^0, \vk_1) \gets \Sh(s)\\
\forall j \in [q]: \qsh_i^j \leftarrow \Ev(C_j, i,
\qsh_i^{j-1})\\
\cert_i \leftarrow \Del(i, \qsh_i^q)
\end{array}
\right] \geq 1 - \negl(\secp)
\end{align}

\item [Compactness:] The following condition holds for all $i \in
\bit$, $q = \poly(\secp)$ and $(C_1, \ldots, C_q) \in
\mathcal{C}^q$, where $l_q$ denotes the output length of the circuit
$C_q$:

\begin{align}
\left[
|\qsh_i^q| - |\qsh_i^{q-1}| = \poly(1^\secp, l_q)
\ :
\begin{array}{ll}
(\qsh_0^0, \vk_0), (\qsh_1^0, \vk_1) \gets \Sh(s)\\
\forall j \in [q]: \qsh_i^j \leftarrow \Ev(C_j, i,
\qsh_i^{j-1})\\
\end{array}
\right]
\end{align}
\end{description}


\subsection{Delete-before-Observe (DbO) Notion}

Let $\calF$ be a function family with all functions having input
domain $\calD$ and co-domain $\calR$. A scheme
satisfying the FSS-DbO syntax for function family $\calF$ is a tuple
of 5 algorithms $\dboFSS = \dboFSS.(\Sh, \Ev, \allowbreak \Obs,
\Del, \Vrfy)$ with the following properties:

\begin{description}

\item [Syntax:] $ $
\item $\Sh(f) \rightarrow (\qf_0^0, \vk_0), (\qf_1^0,
\vk_1)$: The sharing algorithm outputs quantum
(possibly-entangled) function-shares $\qf_0^0, \qf_1^0$ encoding an
input function $f\in\calF$. It also outputs the corresponding
classical verification keys $\vk_0, \vk_1$.

\item $\Ev(x_j, i, \qf_i^{j-1}) \rightarrow \qf_i^{j}$: The
evaluation algorithm takes an input $x_j$, an index $i \in \bit$,
and a share $\qf_i^{j-1}$. It outputs a share $\qf_i^j$.

\item $\Obs(i, \qf_i^j) \rightarrow (y_0^1, \ldots, y_0^j):$ The
observation algorithm takes an index $i \in \bit$ and a quantum
state $\qf_i^j$ and produces a $j$-tuple of classical shares.

\item $\Del(i, \qf_i^j) \rightarrow \cert_i$: The deletion
algorithm takes an index $i \in \bit$, a corresponding quantum share
$\qf_i^j$, and produces a deletion certificate $\cert_i$.

\item $\Vrfy(i, \vk_i, \cert_i) \rightarrow \top/\bot$: The
verification algorithm takes an index $i \in \bit$, the
corresponding verification key $\vk_i$ and a certificate $\cert_i$.
It outputs $\top$ or $\bot$.

\item [Evaluation Correctness:] The following
condition holds for all $q = \poly(\secp)$ and $(x_1, \ldots, x_q)
\in \calD^{q}$:
\begin{align}
\Pr\left[
(y_0^1 \xor y_1^1, \cdots, y_0^q \xor y_1^q) = (f(x_1), \cdots,
f(x_q))
\ :
\begin{array}{ll}
(\qf_0^0, \vk_0), (\qf_1^0, \vk_1) \gets \Sh(f)\\
\forall i,j \in \bit \times [q]: \qf_i^j
\leftarrow \Ev(x_j, i,
\qf_i^{j-1})\\
(y_i^1, \ldots, y_i^q) \leftarrow \Obs(i, \qf_i^q)
\end{array}
\right] \geq 1 - \negl(\secp)
\end{align}

\item [Delete-before-Observe Correctness:] The following
holds for all $i \in \bit$, $q = \poly(\secp)$, and $(x_1, \ldots,
x_q) \in \calD^q:$
\begin{align}
\Pr\left[
\Vrfy(i, \vk_i, \cert_i) \rightarrow \top
\ :
\begin{array}{ll}
(\qf_0^0, \vk_0), (\qf_1^0, \vk_1) \gets \Sh(f)\\
\forall j \in [q]: \qf_i^j \leftarrow \Ev(x_j, i,
\qf_i^{j-1})\\
\cert_i \leftarrow \Del(i, \qf_i^q)
\end{array}
\right] \geq 1 - \negl(\secp)
\end{align}
\end{description}


\subsection{Security Definitions}
\paragraph{Deletion Security wrt Share $j$:}
The following security notion is defined wrt a non-local quantum
adversary $(\qA_0, \qA_1)$:

\begin{description}
\item [$\expa{\dboFSS, (\qA_0, \qA_1)}{del}(1^\secp, j, b)$:] $ $
\begin{enumerate}
\item $\qA_0$ sends $(f_0, f_1) \in \bit^\secp$ to the challenger
$\qCh$.
\item 
$\qCh$ runs
$(\qf_0^0, \vk_0), (\qf_1^0, \vk_1) \gets \Sh(f_b)$ and sends each
$\qf_i^0$ to $\qA_i$.

\item $\qA_j$ sends $(\cert_j, R_j)$ and $\qA_{1-j}$ sends $R_{1-j}$
to $\qCh$, where $R_0, R_1$ are some registers.
\item If $\Vrfy(j, \vk_j, \cert_j) = \top$, then output $(R_0, R_1)$.
\end{enumerate}
\end{description}

\begin{comment}

Statistical Deletion Security wrt Share $j$ holds if the following
holds:

$$TD\Big(\expa{\HSScd, (\qA_0, \qA_1)}{del}(1^\secp, j,
0), \expa{\HSScd, (\qA_0, \qA_1)}{del}(1^\secp, j,
1)\Big) \le \negl(\secp)$$

Computational Deletion Security wrt Share $j$ holds if the following
holds for all QPT $\qA$:

$$\bigg\lvert \Pr\Big[\qA\Big(\expa{\HSScd, (\qA_0,
\qA_1)}{del}(1^\secp, j, 0)\Big)=1\Big] -
\Pr\Big[\qA\Big(\expa{\HSScd, (\qA_0,
\qA_1)}{del}(1^\secp, j, 1)\Big)=1\Big]
\bigg\rvert \le \negl(\secp)$$

\end{description}

\paragraph{Double-Deletion Security:}

The following security notion is defined wrt a non-local quantum
adversary $(\qA_0, \qA_1)$:

\begin{description}
\item [$\expb{\HSScd, (\qA_0, \qA_1)}{del}{2}(1^\secp, b)$:] $ $
\begin{enumerate}
\item $\qA_0$ sends $(s_0, s_1) \in \bit^\secp$ to the challenger.
The challenger runs
$(\qsh_0^0, \vk_0), (\qsh_1^0, \vk_1) \gets \Sh(s_b)$ and sends each
$\qsh_i^0$ to $\qA_i$.

\item $\qA_0$ sends $(\cert_0, R_0)$ and $\qA_{1}$ sends
$(\cert_{1}, R_{1})$ where $R_0, R_1$ are some registers.
\item If $\Vrfy(0, \vk_0, \cert_0) = \Vrfy(1, \vk_1, \cert_1)=
\top$, then output $(R_0, R_1)$.
\end{enumerate}

Statistical Double-Deletion Security holds if the following holds:

$$TD\Big(\expb{\HSScd, (\qA_0, \qA_1)}{del}{2}(1^\secp, 
0), \expb{\HSScd, (\qA_0, \qA_1)}{del}{2}(1^\secp,
1)\Big) \le \negl(\secp)$$

Computational Double-Deletion Security holds if the following holds
for all QPT $\qA$:

$$\bigg\lvert \Pr\Big[\qA\Big(\expb{\HSScd, (\qA_0,
\qA_1)}{del}{2}(1^\secp, 0)\Big)=1\Big] -
\Pr\Big[\qA\Big(\expb{\HSScd, (\qA_0,
\qA_1)}{del}{2}(1^\secp, 1)\Big)=1\Big]
\bigg\rvert \le \negl(\secp)$$

\end{description}

\paragraph{Computational Secrecy wrt Share $j$:}

The following security notion is defined wrt a QPT
adversary $\qA$:

\begin{description}
\item [$\expa{\HSScd, \qA}{ind}(1^\secp, j, b)$:] $ $
\begin{enumerate}
\item $\qA$ sends $(s_0, s_1) \in \bit^\secp$
to the challenger. The challenger runs $(\qsh_0^0, \vk_0),
(\qsh_1^0, \vk_1) \leftarrow \Sh(s_b)$ and sends $\qsh_i^0$ to
$\qA$.
\item
$\qA$ sends $b'$ to the challenger. The challenger outputs $b'$.
\end{enumerate}

$$\bigg\lvert \Pr\Big[\expa{\HSScd, \qA}{ind}(1^\secp, j, 0)=1\Big]
-
\Pr\Big[\expa{\HSScd, \qA}{ind}(1^\secp, j, 1)=1\Big]
\bigg\rvert \le \negl(\secp)$$
\end{description}

Hereafter, we will use \emph{stat} to denote statistical
security and \emph{comp} to denote computational security.

\begin{definition}[(Additive Strong) (X, Y)-HSS-CD scheme for
$\calC$]
An (Additive Strong) (X, Y)-HSS-CD scheme for $\calC$, where X, Y
$\in
\{\textrm{stat},
\textrm{comp}\}^2$  is a scheme that satisfies the (Additive Strong)
HSS-CD
syntax for $\calC$, the X deletion security for share $0$, and the
Y deletion security for share $1$.
\end{definition}

\begin{definition}[(Additive Strong) (X)-HSS-CD scheme for $\calC$] An
(Additive Strong) (X)-HSS-CD scheme for $\calC$ where $X \in
\{\textrm{stat}, \textrm{comp}\}$ is a scheme satisfying the
(Additive Strong) HSS-CD syntax for $\calC$, the X double-deletion
security, and computational secrecy wrt share $0$ and share $1$.
\end{definition}

\begin{remark}
Notice that Deletion Security wrt Share
$j$ implies Computational Secrecy wrt Share $1-j$, but the Double
Deletion security does not imply Computational Secrecy.
\end{remark}

\begin{remark}
Observe that a (stat, comp)-HSS-CD scheme for $\calC$ is also
a (stat)-HSS-CD scheme for $\calC$. Likewise, a (comp,
comp)-HSS-CD scheme for $\calC$ is also a (comp)-HSS-CD scheme for
$\calC$.
\end{remark}
\end{comment}

\cite{}
