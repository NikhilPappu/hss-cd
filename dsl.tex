\section{Definitions of Distributed Software Leasing}\label{sec:dsl}

\subsection{Syntax of DSL}

A scheme satisfying the DSL syntax for a PPT computable circuit
class $\qC$ is a tuple of 4 algorithms $\DSL = \DSL.(\Sh, \Ev, \Del,
\Vrfy)$ with the following properties:

\begin{description}

\item [Syntax:] $ $
\item $\Sh(P) \rightarrow (\qP_0, \vk_0), (\qP_1,
\vk_1)$: The sharing algorithm outputs quantum
program-shares $\qP_0, \qP_1$ encoding an input program $P \in \qC$.
It also outputs the corresponding classical verification keys
$\vk_0, \vk_1$.

\item $\Ev(i, \qP_i, x) \rightarrow y_i$:
The evaluation algorithm takes an index $i \in \bit$, a quantum
program-share $\qP_i$, and a classical input $x$. It
outputs a classical evaluation-share $y_i$.

\item $\Del(i, \qP_i) \rightarrow \cert_i$: The deletion
algorithm takes an index $i \in \bit$, a corresponding program-share
$\qP_i$, and produces a deletion certificate $\cert_i$.

\item $\Vrfy(i, \vk_i, \cert_i) \rightarrow \top/\bot$: The 
classical verification algorithm takes an index $i \in \bit$, the
corresponding verification key $\vk_i$ and a certificate $\cert_i$.
It outputs $\top$ or $\bot$.

\item [Deletion Correctness:] The following condition holds for each
index $i \in \bit$:
\begin{align}
\Pr\left[
\Vrfy(i, \vk_i, \cert_i) \rightarrow \top
\ :
\begin{array}{ll}
(\qP_0, \vk_0), (\qP_1, \vk_1) \gets \Sh(P)\\
\cert_i \leftarrow \Del(i, \qP_i)
\end{array}
\right] \geq 1 - \negl(\secp)
\end{align}

\item [Evaluation Correctness:] The following holds for every input
$x$ in the domain of $P$:

\begin{align}
\Pr\left[
y_0 \xor y_1 = P(x)
\ :
\begin{array}{ll}
(\qP_0, \vk_0), (\qP_1, \vk_1) \gets \Sh(P)\\
y_i \leftarrow \Ev(i, \qP_i, x)
\end{array}
\right] \geq 1 - \negl(\secp)
\end{align}
\end{description}

\begin{remark}
Notice that a correct $y_i$ can be obtained from $\qP_i$ with
overwhelming probability. Hence, from the gentle measurement lemma,
such a $y_i$ can be obtained without almost disturbing the state.
This implies that an arbitrary polynomial number of evaluations can
be performed using $\qP_i$.
\end{remark}

\subsection{Security Definitions}

\begin{definition}[Finite-Term Security]
The following security notion is defined wrt a non-local QPT $\qA =
(\qA_0, \qA_1, \qA_2)$, and a distribution $D_\qC$ over $\qC$:

\begin{description}
\item [$\expb{\DSL, \qA}{fin}{lessor}(1^\secp, D_\qC)$:] $ $
\begin{enumerate}
\item The challenger $\qCh$ samples $P \leftarrow D_\qC$, computes
$(\qP_0, \vk_0), (\qP_1, \vk_1) \gets \Sh(P)$ and sends $\qP_0,
\qP_1$ to $\qA_0, \qA_1$ respectively.

\item $\qCh$ receives $(\cert_0, \rho_0), (\cert_1, \rho_1)$ from $\qA_0,
\qA_1$ respectively.

\item If $\Vrfy(i, \vk_i, \cert_i) = \top$ for some $i \in \bit$,
$\qCh$ samples $x^* \gets \bit^\secp$ and sends $(x^*, \rho_0, \rho_1)$ to
$\qA_2$. Otherwise, it outputs $\bot$.

\item $\qA_2$ sends $y^*$ to $\qCh$. If $y^* = P(x^*)$, $\qCh$
outputs $\top$. Else, it outputs $\bot$.

\end{enumerate}

\end{description}

A DSL scheme $\DSL$ has deletion security if for all QPT adversaries
$\qA$, the following holds:

$$\Pr\Big[\expb{\DSL, \qA}{fin}{lessor}(1^\secp, D_\qC) \ra \top\Big]
\le \negl(\secp)$$
\end{definition}

\begin{definition}[Secrecy]

The following security notion is defined wrt a QPT
adversary $\qA$:

\begin{description}
\item [$\expa{\DSL, \qA}{ind}(1^\secp, b)$:] $ $
\begin{enumerate}
\item $\qA$ sends $(P^0, P^1)$ and $i \in \bit$
to $\qCh$. $\qCh$ runs $(\qP_0, \vk_0),
(\qP_1, \vk_1) \leftarrow \Sh(P^b)$ and sends $\qP_i$ to $\qA$.
\item
$\qA$ sends $b'$ to $\qCh$. $\qCh$ outputs $b'$ as the final
output.
\end{enumerate}

$$\bigg\lvert \Pr\Big[\expa{\DSL, \qA}{ind}(1^\secp, 0)=1\Big]
-
\Pr\Big[\expa{\DSL, \qA}{ind}(1^\secp, 1)=1\Big]
\bigg\rvert \le \negl(\secp)$$
\end{description}
\end{definition}

\begin{comment}
\begin{definition}[Strong Deletion Security]
TBD. ($\qA_2$ learns nothing more than having black-box access
to $P$).
\end{definition}

\begin{definition}[Secrecy]

The following security notion is defined wrt a QPT adversary $\qA$:

\begin{description}
\item [$\expa{\DSL, \qA}{ind}(1^\secp, \coin)$:] $ $
\begin{enumerate}
\item $\qA$ sends $(P^0, P^1) \in \qC^2$ and $b \in \bit$
to the challenger $\qCh$. $\qCh$ runs $(\qP_0, \vk_0),
(\qP_1, \vk_1) \leftarrow \Sh(P^\coin)$ and sends $\qP_b$ to
$\qA$.
\item $\qA$ sends $\coin'$ to $\qCh$ which is the final output of
the experiment.
\end{enumerate}
\end{description}

A DSL scheme $\DSL$ has secrecy if the following notion holds for
all QPT adversaries $\qA$:

$$\bigg\lvert \Pr\Big[\expa{\DSL, \qA}{ind}(1^\secp, 0)=1\Big]
-
\Pr\Big[\expa{\DSL, \qA}{ind}(1^\secp, 1)=1\Big]
\bigg\rvert \le \negl(\secp)$$

\end{definition}
\end{comment}

\section{Inapplicability of SSL Impossibility}

We will now briefly review the impossibility result regarding SSL
by Ananth and LaPlaca \cite{EC:AnaLaP21}. They showed the existence
of a class of unlearnable circuits that cannot be securely-leased
(and thereby copy-protected). The corresponding circuit class $\qC$
consists of circuits of the form $C_{a, b, r, \pk, \calO}(x)$,
described as follows:

\begin{description}
\item[\underline{$C_{a, b, r, \pk, \calO}(x)$:}] $ $
\begin{enumerate}
\item If $x = 0 \ldots 0$, output $\QFHE.\qEnc(\pk, a; r) \| \calO
\| \pk$.
\item Else, if $x = a$, output $b$.
\item Otherwise, output $0 \ldots 0$.
\end{enumerate}
\end{description}

The corresponding description $D_\qC$ samples a circuit $C$ from
$\qC$ as follows:
\begin{description}
\item[\underline{$D_\qC$:}] $ $
\begin{enumerate}
\item Sample $a, b, r \gets \bit^\secp$.
\item Compute $(\pk,\sk,\rho_\evk) \gets \QFHE.\qGen(1^\secp)$.
\item Compute $\calO \gets \Obf \big(\QFHE.\qDec(\sk, \cdot), b, 
\sk \| r\big)$.
\item Output $C_{a, b, r, \pk, \calO}$.
\end{enumerate}
\end{description}

It was shown in \cite{EC:AnaLaP21} that given (quantum) black-box
access to $C$ for $C \gets D_\qC$, it is infeasible to learn the
description of the circuit. However, given access to a quantum
implementation $(U_C, \rho_C)$ of $C \in \qC$, there exists an
algorithm $\qB$ that learns the description of $C$. Since this
description is classical, it can be copied. The algorithm $\qB$ is
described as follows:

\begin{description}
\item[\underline{$\qB(U_C, \rho_C)$}:] $ $
\begin{enumerate}
\item Compute $(\rho', \ct_1 \| \calO'\| \pk') \gets U_C(\rho_C,
0\ldots0)$.
\item Compute $\sigma \| \ct_2 \gets \QFHE.\Eval(U_C(\rho', \cdot),
\ct_1)$.
\item Compute $\sk' \| r \gets \calO(\ct_2)$.
\item $a' \gets \QFHE.\Dec(\sk', \ct_1), b' \gets \QFHE.\Dec(\sk',
\ct_2)$.
\item Compute $a' \gets \QFHE.\Dec(\sk', \ct_1), b' \gets
\QFHE.\Dec(\sk', \ct_2)$.
\item Output $C_{a', b', r', \pk', \calO'}$.
\end{enumerate}
\end{description}

It is trivially true that $\pk' = \pk$ and $\calO' = \calO$. Let us
observe why $\qB$ succeeds in obtaining $a' = a$ and $b' = b$. After
Step 1., the state $\rho'$ is close to the state $\rho_C$ by
correctness of evaluation. From the evaluation correctness of
$\QFHE$, $\ct_2$ is a valid encryption of $b$. Therefore, $\sk' =
\sk$ is output by the lockable obfuscation circuit $\calO$. Finally,
$\sk$ is used to decrypt $\ct_1, \ct_2$ to obtain $a' = a$ and
$b'=b$ respectively. Furthermore, it was shown in their work that
$\sigma$ can be used to obtain a state $\rho''$ close to the state
$\rho_C$ which can then be deleted.

Observe now that an attack of the above kind does not work in the
distributed setting. This is because a quantum share $\qP_i$ of a
circuit $P \in \qC$ would only reveal an additive share of the FHE
ciphertext $\ct_1$ in Step 1. Hence, it is unclear if the
description of $\qP_i$ (or $P$) can be learnt before deletion, even
in a way that the descriptions are only revealed after collusion.

Upon looking into it further, it seems like it should be impossible,
at least for Encrypted Software Leasing (ESL). If the adversary could
do a malleability attack, then that is enough. To show such an attack
always exists is much harder than in the nice QFHE case though. One thing is
that a malleability attack should exist (which is a classical crypto
question) but we also need the fact that the attack doesn't disturb
the state.

\begin{comment}
\begin{remark}[Necessity of Secrecy]
Do we need secrecy in the first stage to beat SSL impossibility?
\end{remark}
\end{comment}

\section{DSL for Puncturable Functions}

\begin{itemize}
\item Try to construct from dIO-CD + PKE (similar to FSS
construction).

\item For the dIO-CD + PKE construction, we need to have a distributed
diO. $\qA_0$ receives one copy, $\qA_1$ receives another (note that we
don't care about the secret keys $\sk_0, \sk_1$ here as we can assume
both have both unlike in the construction). $\qA_0$ deletes and
$\qA_1$ deletes and they combine, so it is even weaker than
2-collusion-resistant security.
\end{itemize}

\section{Distributed One-Time Programs}

One time programs allow a software to be encoded into a format that
enables only a single evaluation. Unfortunately, deterministic programs
cannot be one-time protected. In this work, we explore the setting
where a deterministic program can be split between two parties. In
this setting, we explore whether it is possible to enforce only a single
distributed evaluation. We show a feasibility where the program shares
are entangled and prove that such entanglement is necessary.\\

\noindent
\textbf{Use Case:} A party can
distribute a program to two servers who are supposed to execute the
software within a short period. For proof, they can post commitments
to their shares online or on the blockchain. During this time, one
might be able to enforce that the servers do not directly collude with
each other due to the fear of apprehension. However, they might
eventually sell their data to random entities on the black market.
This notion guarantees that if the initial evaluation was correctly performed
in a distributed fashion, nothing can be done to evaluate a second time.

\subsection{Definition}

A Distributed One-Time Program (D-OTP) scheme $\DOTP =
(\KG,\TG,\TE)$ has the following syntax:

\begin{description}

\item [Syntax:] $ $
\item $\KG(f) \rightarrow (\tP_0, \tP_1, \sk)$: The key-generation algorithm
outputs two classical programs $\tP_0, \tP_1$ and a secret-key $\sk$.

\item $\TG(\sk) \ra (\qtk_0, \qtk_1)$: The token-generation algorithm takes
in a secret-key and generates (possibly-entangled) quantum tokens
$\qtk_0, \qtk_1$.

\item $\TE(x, \qtk_i, \tP_i) \rightarrow y_i$:
The token-evaluation algorithm takes a token $\qtk_i$, a
program $\tP_i$, and a classical input $x$. It
outputs a classical evaluation-share $y_i$.

\item [Evaluation Correctness:] The following holds for every input
$x$ in the domain of $f$:

\begin{align}
\Pr\left[
y_0 \xor y_1 = f(x)
\ :
\begin{array}{ll}
(\tP_0, \tP_1, \sk) \leftarrow \KG(f)\\
(\qtk_0, \qtk_1) \gets \TG(\sk)\\
\forall i\in\bit: y_i \leftarrow \TE(x, \qtk_i, \tP_i)
\end{array}
\right] \geq 1 - \negl(\secp)
\end{align}
\end{description}

\begin{remark}
Due to the gentle-measurement lemma, the evaluation correctness
implies that the state described by $\qtk_0, \qtk_1$ can be evaluated
on without almost disturbing the state. This seems to contradict any
meaningful notion of one-time security. However, our distributed
notion will make use of the fact that the non-local adversary must
produce a correct distributed evaluation first before collusion is
allowed. Since the tokens $\qtk_0, \qtk_1$ can be entangled, $y_0$ and
$y_1$ can be truly random in a local-sense (even for a fixed global
token state).
\end{remark}


\begin{definition}[Distributed One-Time Security]
Consider the following experiment between a challenger $\qCh$ and a
QPT non-local adversary $\qA = (\qA_0, \qA_1, \qA_2)$ for the scheme
$\DOTP = (\KG, \TG, \TE)$ and a function $f$ sampled from a
distribution $\cD$:

\begin{description}

\item $\underline{\expb{\DOTP, \qA}{d}{ots}}(1^\secp, f, \cD)$:
\begin{enumerate}
\item $\qCh$ generates $f \gets \cD$, $(\tP_0, \tP_1, \sk) \gets
\KG(f)$ and
$(\qtk_0, \qtk_1) \gets \TG(\sk)$.
\item $\qCh$ sends $(\qtk_i, \tP_i)$ to adversary $\qA_i$ for each $i
\in \bit$.
\item For $i \in \bit$, if adversary $\qA_i$ sends a register $R_i$,
run $\tP_i$ on it and measure the output register to get outcome
$y_i$. Send the register $R_i$ to adversary $\qA_2$.
\item Define $y \seteq y_0 \xor y_1$.
\item For each $i \in \bit$, wait to receive register $R_i'$ from
$\qA_2$. Then, run $\tP_i$ on it and measure the output register
to get outcome $y_i'$.

\item Define $y' \seteq y_0' \xor y_1'$.

\item If $y' \neq y$, output $\top$. Else, output $\bot$.
\end{enumerate}
\end{description}

We require the following condition to hold:


$$\Pr\Big[\expb{\DOTP, \qA}{d}{ots}(1^\secp, f, \cD) \ra \top\Big]
\le \negl(\secp)$$

\end{definition}

Notice that the above definition allows collusion after the first
distributed evaluation. As per the definition, $f$ can even be a
deterministic function. It remains to be seen which functions can be
handled. At the very least, $f$ must be unlearnable given a single
black-box query, as the ``collusion-stage" adversary $\qA_2$ can
compute the first evaluation $y$. The quintessential example would be
a PRF family $\{F_k\}_k$, i.e. $P = F_k$ for $k \gets \bit^\secp$
which is clearly unlearnable.

\begin{comment}
\begin{remark}
We will also consider a weaker notion $\expc{\DOTP,\qA}{d}{ots}{weak}$
where $R_0, R_1$ are sent back to $\qA_0, \qA_1$ who then send $R_0',
R_1'$. In other words, collusion is never allowed in this variant, so
$\qA_2$ is not part of the experiment.
\end{remark}
\end{comment}

\subsection{Local One-Time Authentication}

A Local One-Time Authentication (LOTA) scheme is a tuple of 5
algorithms $\LOTA = (\KG, \TG, \qSign, \Vrfy)$ with the
following syntax:

\begin{description}
\item $\KG(1^\secp) \ra \sk$: The key-generation algorithm outputs a
classical secret-key $\sk$.
\item $\TG(\sk) \ra (\qtk_0, \qtk_1)$: The token-generation algorithm
takes the secret-key $\sk$ as input. It outputs (possibly-entangled)
quantum token states $(\qtk_0, \qtk_1)$.
\item $\qSign(x, \qtk_i) \ra z$: The signing algorithm takes as
input a token state $\qtk_i$ and a message $x$. It outputs a signature $z$.
\item $\Vrfy(\sk, (x, z)) \ra \top/\bot$ The verification algorithm
$\Vrfy$ takes as input the secret-key $\sk$ and a message-signature
pair $(x, z)$. It outputs $\top$ or $\bot$.

\item[Same Signatures:] The following holds for all $x \in \bit$:

\begin{align}
\Pr\left[
z_0 = z_1
\ :
\begin{array}{ll}
\sk \gets \KG(1^\secp)\\
(\qtk_0, \qtk_1) \gets \TG(\sk)\\
z_0 \leftarrow \qSign(x, \qtk_0)\\
z_1 \leftarrow \qSign(x, \qtk_1)
\end{array}
\right] \geq 1 - \negl(\secp)
\end{align}

\item[Verification Correctness:] The following holds for all
$x\in\bit$ and $i \in \bit$:

\begin{align}
\Pr\left[
\Vrfy(\sk, (x,z)) \ra \top
\ :
\begin{array}{ll}
\sk \gets \KG(1^\secp)\\
(\qtk_0, \qtk_1) \gets \TG(\sk)\\
z \leftarrow \qSign(x, \qtk_i)
\end{array}
\right] \geq 1 - \negl(\secp)
\end{align}


\end{description}

\begin{definition}[One-Time Authentication]
Consider the following experiment $\expa{\LOTA,
\qA}{ota}(1^\secp)$ between a challenger $\qCh$ and a QPT adversary
$\qA$:
\begin{description}
\item \underline{$\expa{\LOTA, \qA}{ota}(1^\secp)$}:
\begin{enumerate}
\item $\qCh$ runs $\sk \gets \KG(1^\secp)$ and
$(\qtk_0, \qtk_1) \gets \TG(\sk)$.
\item $\qCh$ sends $\qtk_0$ to $\qA$.
\item $\qA$ sends $(z, \widetilde{z})$ to $\qCh$.
\item If $\Vrfy(\sk, (0, z)) = \Vrfy(\sk, (1, \widetilde{z})) =
\top$, $\qCh$ outputs $\top$. Otherwise, it output $\bot$.
\end{enumerate}
\end{description}

We require that the following condition holds:

$$\Pr[\expa{\LOTA, \qA}{ota}(1^\secp) = \top] \le \negl(\secp)$$
\end{definition}


\subsection{Construction in the Oracle Model}

We will use the following idealized oracles:
\begin{itemize}
\item Random Oracle $H$.
\item VBB Obfuscation $\Obf$.
\item Local One-Time Authentication Scheme $\LOTA = \LOTA.(\KG, \TG,
    \qSign,
\Vrfy)$.
\end{itemize}

\begin{description}
\item $\underline{\KG(f)}:$
\begin{itemize}
\item Sample $\sk \gets \LOTA.\KG(1^\secp)$.
\item Let $P$ be the following program and $\tP \seteq \Obf(P)$.
\begin{description}
\item $\underline{P_0 (x, z)}:$
\begin{itemize}
\item If $\Vrfy(\sk, (x, z)) = \bot$, output $\bot$.
\item Otherwise, output $P(x) \xor H(z)$.
\end{itemize}
\end{description}

\item Output $(\tP_0, \tP_1, \sk)$.
\end{itemize}

\item $\underline{\TG(\sk)}:$
\begin{itemize}
\item Compute $(\qtk_0, \qtk_1) \gets \LOTA.\TG(\sk)$.
\item Output $(\qtk_0, \qtk_1)$.
\end{itemize}

\item $\underline{\TE(x, \qtk_i, \tP_i)}:$
\begin{itemize}
\item Compute $z \gets \qSign(x, \qtk_i)$.
\item Compute and output $y_i \seteq \tP_i(x, z)$.
\end{itemize}


\end{description}
